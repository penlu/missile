\documentclass{article}

% Load basic packages
\usepackage{balance}
%\usepackage{graphics}
%\usepackage{txfonts}
%\usepackage{mathptmx}
%\usepackage{enumitem,gensymb}
%\usepackage{color}
%\usepackage{booktabs}
%\usepackage{subfigure}
%\usepackage{textcomp}
%\usepackage{float}
\usepackage{xspace}
%\usepackage{mathpartir}
\usepackage{amsmath}
%\usepackage{amssymb}
%\usepackage{multirow}
%\usepackage{siunitx}
\usepackage{algorithm}
\usepackage{algpseudocode}
\usepackage{listings}
\usepackage{tikz}
\usepackage{caption}
\usepackage{subcaption}
\usepackage{physics}

\DeclareMathOperator*{\argmax}{arg\,max}

\lstdefinestyle{style}{
  basicstyle=\ttfamily\small,
  numbers=left,
}
\lstset{style=style}

% author notes (Justin)

\newcommand{\authnote}[3]{\textcolor{#3}{[{\footnotesize {\bf #1:} { {#2}}}]}}
\newcommand{\jnote}[1]{\authnote{J}{#1}{blue} }
\newcommand{\enote}[1]{\authnote{J}{#1}{green} }

% some convenience commands
\newcommand{\TODO}[1]{{\bf\color{red} #1}}
\newcommand{\OUTLINE}[1]{{\color{blue} #1}}

% reduce indentation in algorithmic
\algrenewcommand\algorithmicindent{1em}

% add goto to algorithmic
\newcommand{\algorithmautorefname}{Algorithm}
\algnewcommand{\algorithmicgoto}{\textbf{go to}}%
\algnewcommand{\Goto}[1]{\algorithmicgoto~\ref{#1}}%

% add switch/case
\algnewcommand\algorithmicswitch{\textbf{switch}}
\algnewcommand\algorithmiccase{\textbf{case}}
\algnewcommand\algorithmicassert{\texttt{assert}}
\algnewcommand\Assert[1]{\State \algorithmicassert(#1)}%
\algdef{SE}[SWITCH]{Switch}{EndSwitch}[1]{\algorithmicswitch\ #1\ \algorithmicdo}{\algorithmicend\ \algorithmicswitch}%
\algdef{SE}[CASE]{Case}{EndCase}[1]{\algorithmiccase\ #1}{\algorithmicend\ \algorithmiccase}%
\algtext*{EndSwitch}%
\algtext*{EndCase}%

\def\pprw{8.5in}
\def\pprh{11in}
\special{papersize=\pprw,\pprh}
\setlength{\paperwidth}{\pprw}
\setlength{\paperheight}{\pprh}
\setlength{\pdfpagewidth}{\pprw}
\setlength{\pdfpageheight}{\pprh}

\newcommand{\technique}[0]{\textsc{Algorithm~E}\xspace}
\newcommand{\SSE}[0]{S\textsuperscript{2}E\xspace}
\newcommand{\KLEE}[0]{\textsc{Klee}\xspace}
\newcommand{\angr}[0]{\textsc{Angr}\xspace}

\frenchspacing
\begin{document}

\title{Optimal pursuit by a missile of an evading spacecraft}
\author{penlu}

\maketitle

\section{Introduction}

A missile and a spacecraft exist in space with some specified distance and
relative velocity.  The missile's maximum acceleration is $M$ times that of the
spacecraft.  What is the minimum $\Delta v$ with which the missile can assuredly
impact an optimally evading spacecraft?  What do the optimal pursuit and evasion
trajectories look like?

\section{Dimensionality reduction}

The problem is rotationally symmetric. Moreover, the distance can be set to 1
wlog via a change of units. This means that in the simple case, the problem only
has two free parameters, which are the missile velocity in a plane.

TODO breaking things down into $(\rho, \sigma, \tau)$ the distance, radial
velocity, and tangential velocity.

\jnote{Added:}
\renewcommand{\vec}{\mathbf}
Call the pursuer $A$ and the evader $B$.  Let $\vec{x}_A$
  denote the position of the
  pursuer, and $\vec{x}_B$ the position of the evader.  Define $\vec{r} = \vec{x}_B
  - \vec{x}_A$, $r = | \vec{r} |$, $\vec{v} = \big | \dot{\vec{x}}_B -
  \dot{\vec{x}}_A \big |$, $v = | \vec{v} |$, and $\theta = \arccos \big ( \vec{r}
  \cdot \vec{v} \big )$.

  It is clear that (along with the constant scale parameter
  $a_{\max}$),$(r, v, \theta)$ completely describes the game state.  However, we
  can go one step further in dimensionality reduction.  By Buckingham's $\pi$
  theorem, any physically meaningful quantity can depend only on the
  dimensionless quantities that can be derived from $(r, v, \theta, a_{\max})$,
  of which there are only two, namely $\kappa := \frac{v^2}{r \cdot a_{\max}}$
  and $\theta$.

  We care about the following unitless quantities:
  \begin{enumerate}
  \item The magnitude of the missile's acceleration, normalized to a unitless
    quantity by dividing by $a_{\max}$.
  \item The direction of the missile's acceleration.
  \item The minimax $\Delta v$ that the missile needs to `catch' the spaceship,
    normalized to a unitless quantity by dividing by $\sqrt{r \cdot a_{\max}}$.\footnote{Note that we
      might instead normalize by dividing by $v$, but that introduces the
      possibility that $v$ is $0$, and anyways is equivalent up to division by $\sqrt{\kappa}$.}
  \end{enumerate}


TODO drawing a picture

\section{Formulation as a differential game}

We formulate the problem as a differential game and employ the
Hamilton--Jacobi--Isaacs (HJI) equation to obtain a differential equation constraining
the value function.

\subsection{The HJI equation}

Let us introduce some notation:
\begin{enumerate}
\item $\vec{s}$ is a vector representing the game state, whose elements are
$(s_1, \dots, s_n)$
\item $t$ is time
\item $\phi$ is the pursuer's control
\item $\psi$ is the evader's control
\item $\nabla$ is the vector of partial derivatives $\left(\pdv{}{s_1}, \dots,
\pdv{}{s_n}\right)$
\item $V(\vec{s}, t)$ is the value function of the game
\item $F(\vec{s}, t, \phi, \psi)$ is a function representing the system dynamics
\item $G(\vec{s}, t, \phi, \psi)$ is the cost rate function
\item $H(\vec{s}, t)$ is the bequest value
\end{enumerate}

$F$ represents the dynamics of the system via the first-order differential
equation:
\begin{equation}
\dot{\vec{s}} = F(\vec{s}, \phi, \psi)
\end{equation}

The HJI equation may be written as follows:
\begin{align}
\pdv{V(\vec{s}, t)}{t} + \min_{\phi} \max_{\psi} \left[
  \nabla V(\vec{s}, t) \cdot F(\vec{s}, \phi, \psi) + G(\vec{s}, \phi, \psi)
\right]
\end{align}
subject to the boundary condition $V(\vec{s}, t) = H(\vec{s}, t)$ within the
capture set.
TODO more considerations for capture formalism may exist in general?

\subsection{Specializing to the 1M1S game}

We specialize the HJI equation for the case of the missile pursuit game.
Here, $V$, $F$, and $G$ lack time dependence, and $H$ is zero.

As previously established, the state space $\vec{s}$ is three-dimensional:
\begin{align}
\vec{s} = (\rho, \sigma, \tau)
\end{align}
where $\rho$ is the distance, $\sigma$ is the radial velocity, and $\tau$ is the
tangential velocity.

The system dynamics are:
\begin{align}
\dot{\vec{s}}
&= F(\rho, \sigma, \tau, \vec{a}_m, \vec{a}_s) \notag \\
&= \left(\sigma,
  \frac{\tau^2}{\rho} + a_{m\sigma} - a_{s\sigma},
  -\frac{\sigma\tau}{\rho} + a_{m\tau} - a_{s\tau}\right)
\end{align}

We will also write $\vec{a}_m = \phi$ for the missile's control and $\vec{a}_s =
\psi$ for the spacecraft's control.

Substituting all of this, the HJI equation for this game is:
\begin{align}
\min_{\vec{a}_m} \max_{\vec{a}_s} \left[
  \pdv{V}{\rho} \sigma
+ \pdv{V}{\sigma} \left(\frac{\tau^2}{\rho} + a_{m\sigma} - a_{s\sigma}\right)
+ \pdv{V}{\tau} \left(-\frac{\sigma\tau}{\rho} + a_{m\tau} - a_{s\tau}\right)
+ ||\vec{a}_m||
\right] = 0
\end{align}

I am not really sure how to solve something that looks like that.
The reason the existence of $\pdv{V}{x}$ terms is problematic for a numerical
solution is because $\dot{x}$ can be very high: the missile acceleration may be
effectively infinite.
Then the Taylor series approximation used in the HJB equation derivation breaks
down.
That is, in our case, $V(x(t + dt))$ is not well-approximated as $V(x) +
\pdv{V}{x} \dot{x} dt$.
Fortunately, we can actually work with $V(x(t + dt))$ directly, using Bellman's
original \textit{discrete} principle of optimality.

The idea is that for a given state $\vec{s}$, the value after a single timestep
had better be the minmax value over states accessible after a single timestep.
\begin{align}
V(\vec{s}) = \min_{\vec{a}_m} \max_{\vec{a}_s} \left[
  V(\vec{s}_\text{next}(\vec{a}_m, \vec{a}_t, dt)) +
  ||\vec{a}_m||
\right]
\end{align}

Here, $\vec{s}_\text{next}$ is basically just $\vec{s} + F(\dots) dt$.

\section{Analysis of acceleration matching}

Acceleration matching is an intuitive strategy that has been proposed and
analyzed previously. Here we recapitulate the analysis. We repeat the derivation
of the optimal allocation of $\Delta$v between the boost and terminal stages,
and the maximum effective range of this missile assuming a $\Delta$v limit.

In the acceleration matching strategy, missile behavior has two stages:
\begin{enumerate}
\item Boost phase, in which the missile instantaneously gains some velocity
$v_\text{boost}$ toward the target.
\item Terminal phase, in which the missile accelerates to match target
accelerations. In this stage the missile has some $\Delta$v budget
$v_\text{term}$.
\end{enumerate}

Consider the situation where the missile and target begin at time $t = 0$ at a
distance $D$ apart and 0 relative velocity. Assume the target has an
acceleration capacity $a$.

In acceleration matching, during the terminal phase, the missile maintains a
constant closure velocity $v_\text{boost}$, since it thrusts to cancel the
target acceleration. The time to reach the target $t_\text{hit}$ is given by:
\begin{equation}
t_\text{hit} = \frac{D}{v_\text{boost}}
\end{equation}

The missile is defeated if it runs out of $\Delta$v before reaching the target.
If the target maintains full acceleration $a$ in any direction, the missile
$\Delta$v budget runs out in time $t_\text{defeat}$ given by:
\begin{equation}
t_\text{defeat} = \frac{v_\text{term}}{a}
\end{equation}

In order to hit, we require $t_\text{hit} \le t_\text{defeat}$. Using this, we
can derive the maximum distance $D$ at which the missile may hit the target
given $v_\text{boost}$, $v_\text{term}$, and $a$:
\begin{align}
\frac{D}{v_\text{boost}} &\le \frac{v_\text{term}}{a} \\ \notag
D &\le \frac{v_\text{term} v_\text{boost}}{a}
\end{align}

Given a total $\Delta$v budget $v = v_\text{boost} + v_\text{term}$, we may
select an apportionment that maximizes $D$ by maximizing a quadratic:
\begin{align}
v_\text{boost}^* &= \argmax_{v_\text{boost}}
    \frac{v_\text{boost} (v - v_\text{boost})}{a} \\
v_\text{boost}^* &= \frac{v}{2}
\end{align}

Substituting, we obtain the maximum hit distance $D^*$ for a given total
$\Delta$v budget $v$:
\begin{equation}
D^* = \frac{v^2}{4a}
\end{equation}

\section{Lateral acceleration matching is suboptimal}

We give a simple argument that accelerating longitudinally in the terminal phase
leads to a longer effective range.

\balance

\newpage

\bibliography{paper}

\end{document}

%%% Local Variables:
%%% mode: latex
%%% TeX-master: t
%%% End:
