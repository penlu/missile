\documentclass{article}
% Load basic packages
\usepackage{balance}
\usepackage{fullpage}
%\usepackage{graphics}
%\usepackage{txfonts}
%\usepackage{mathptmx}
%\usepackage{enumitem,gensymb}
%\usepackage{color}
%\usepackage{booktabs}
%\usepackage{subfigure}
%\usepackage{textcomp}
%\usepackage{float}
\usepackage{xspace}
%\usepackage{mathpartir}
\usepackage{amsmath}
\usepackage{cleveref}
%\usepackage{amssymb}
%\usepackage{multirow}
%\usepackage{siunitx}
\usepackage{algorithm}
\usepackage{algpseudocode}
\usepackage{listings}
\usepackage{tikz}
\usepackage{caption}
\usepackage{subcaption}
\usepackage{physics}
\usepackage{amssymb}

\DeclareMathOperator*{\argmax}{arg\,max}

\lstdefinestyle{style}{
  basicstyle=\ttfamily\small,
  numbers=left,
}
\lstset{style=style}

% author notes (Justin)

\newcommand{\authnote}[3]{\textcolor{#3}{[{\footnotesize {\bf #1:} { {#2}}}]}}
\newcommand{\jnote}[1]{\authnote{J}{#1}{blue} }
\newcommand{\enote}[1]{\authnote{J}{#1}{green} }
\newcommand{\eqdef}{:=}

% some convenience commands
\newcommand{\TODO}[1]{{\bf\color{red} #1}}
\newcommand{\OUTLINE}[1]{{\color{blue} #1}}

% reduce indentation in algorithmic
\algrenewcommand\algorithmicindent{1em}

% add goto to algorithmic
\newcommand{\algorithmautorefname}{Algorithm}
\algnewcommand{\algorithmicgoto}{\textbf{go to}}%
\algnewcommand{\Goto}[1]{\algorithmicgoto~\ref{#1}}%

% add switch/case
\algnewcommand\algorithmicswitch{\textbf{switch}}
\algnewcommand\algorithmiccase{\textbf{case}}
\algnewcommand\algorithmicassert{\texttt{assert}}
\algnewcommand\Assert[1]{\State \algorithmicassert(#1)}%
\algdef{SE}[SWITCH]{Switch}{EndSwitch}[1]{\algorithmicswitch\ #1\ \algorithmicdo}{\algorithmicend\ \algorithmicswitch}%
\algdef{SE}[CASE]{Case}{EndCase}[1]{\algorithmiccase\ #1}{\algorithmicend\ \algorithmiccase}%
\algtext*{EndSwitch}%
\algtext*{EndCase}%

\def\pprw{8.5in}
\def\pprh{11in}
\special{papersize=\pprw,\pprh}
\setlength{\paperwidth}{\pprw}
\setlength{\paperheight}{\pprh}
\setlength{\pdfpagewidth}{\pprw}
\setlength{\pdfpageheight}{\pprh}

\newcommand{\technique}[0]{\textsc{Algorithm~E}\xspace}
\newcommand{\SSE}[0]{S\textsuperscript{2}E\xspace}
\newcommand{\KLEE}[0]{\textsc{Klee}\xspace}
\newcommand{\angr}[0]{\textsc{Angr}\xspace}
\renewcommand{\vec}[1]{\mathbf{#1}}

\frenchspacing

\begin{document}

\title{Optimal pursuit by a missile of an evading spacecraft}
\author{penlu \and grumpyoldman}

\maketitle

\section{Problem Statement}
\newcommand{\R}{\mathbb{R}}

Consider a pursuing missile $P$ and an evading spacecraft $E$ in free space.  We
will consider their future trajectories $\vec{x}_P(t), \vec{x}_E(t) \in \R^3$ for $t \ge 0$,
given initial conditions $\vec{x}_P(0)$, $\vec{x}_E(0)$, $\dot{\vec{x}}_P(0)$, and $\dot{\vec{x}}_E(0)$.

The missile and spacecraft can dynamically choose any twice-differentiable trajectories
subject to the constraints \begin{equation}
  \|
  \ddot{\vec{x}}_P(t) \| \le A
\end{equation}
and
\begin{equation}
    \|
  \ddot{\vec{x}}_E(t) \| \le \beta A,
\end{equation}
where $0 \le \beta < 1$ is a fixed parameter.  To be more precise, the missile
and spacecraft strategies are functions $\vec{a}_P, \vec{a}_E$ that take as
input $\vec{x}_P(t)$, $\vec{x}_E(t)$, $\dot{\vec{x}}_P(t)$,
$\dot{\vec{x}}_E(t)$, and output a vector in $\R^3$ bounded in norm by $A$ or
$ \beta A$.  Once $\vec{a}_P$ and $\vec{a}_E$ are chosen, the trajectories are
then defined as the solution to the ODEs
\begin{align}
  \ddot{\vec{x}}_P(t) &= \vec{a}_P \big (\vec{x}_P(t), \vec{x}_E(t),
                        \dot{\vec{x}}_P(t), \dot{\vec{x}}_E(t) \big )\\
  \ddot{\vec{x}}_E(t) &= \vec{a}_E\big (\vec{x}_P(t), \vec{x}_E(t),
                        \dot{\vec{x}}_P(t), \dot{\vec{x}}_E(t) \big ).
\end{align}

The objectives of the missile, in order of decreasing priority, are:
\begin{enumerate}
\item To achieve $\vec{x}_P(\tau) = \vec{x}_E(\tau)$
for some $\tau \ge 0$.
\item To achieve (1) with minimal ``delta-V''
\begin{equation}
  \label{eq:missile-delta-v}
  \Delta v_P \eqdef \int_{0}^\tau \big \| \ddot{\vec{x}}_P(t) \big \| dt.
\end{equation}
\item To achieve (1) with minimal $\tau$.
\end{enumerate}
The objectives of the spacecraft are exactly the opposite: maximize $\Delta v_P$, and among all
maximizing trajectories, maximize $\tau$. (our assumption
that $\beta < 1$ makes (1) inevitable).

We wish to understand optimal pursuit and evasion strategies, their resulting
trajectories, and their costs.  Dually, we wish to understand the ``missile
capture'' envelope -- the set of initial conditions under which the missile can
assure (1) occurs while subjecting $\Delta v_P$ to a fixed bound.
\section{Missile Max-Burns WLOG}
\jnote{Need to think about this carefully.  I think it should be true that the
  missile always max burns, but I'm not sure.}

\section{Existence of a Finite $\Delta v$ Strategy}
We claim that even if the missile must commit to its acceleration before the
spaceship chooses its acceleration, the missile has a winning strategy whenever
$\beta < 1$.  \jnote{todo: to see this, note that the missile can control the
  relative velocities (disk of possible net relative accelerations excludes zero
  in a direction of missile's choice).  Excess acceleration can then be used to
  produce closure.}

\section{Existence of a Pure Strategy Nash Equilibrium}
In this section we observe that in any game state, there is a (unique) equilibrium
of accelerations, wherein neither missile nor spaceship are incentivized to
change.  Specifically, the spaceship will accelerate as hard as it can in the
direction of $\frac{\partial V}{\partial \dot{\vec{x}}}$, and the missile will
either:
\begin{itemize}
\item
accelerate as hard as it can in the diametrically opposite direction, if
$\norm{\frac{\partial V}{\partial \dot{\vec{x}}}} > 1$;
\item not accelerate at all if $\norm{\frac{\partial V}{\partial \dot{\vec{x}}}}
  < 1$;
\item accelerate at any (non-negative) amount in the direction of
  $-\frac{\partial V}{\partial \dot{\vec{x}}}$ if
  $\norm{\frac{\partial V}{\partial \dot{\vec{x}}}} = 1$.
\end{itemize}

\section{Game Value Differentiability}
In this section we argue that $V$ is differentiable at any non-zero distance.
\jnote{todo: continuity at least  should follow from scaling down the fact that a missile can
  control relative velocity with finite $\delta v$ and finite time, to
  infinitesimally different states.  differentiability should be similar}

\section{Defining Equations for the Game Value}

$V$ should be fully determined by the following equations:
\begin{align*}
  V(\vec{0}, \dot{\vec{r}}) = 0  \text{ for all $\dot{\vec{r}}$}
\end{align*}
and
\[
  \frac{\partial V}{\partial \vec{q}} \cdot \dot{\vec{q}} + (1 - \beta) \cdot
  \left (\norm{\frac{\partial V}{\partial \dot{\vec{q}}}} - 1 \right) = 0
\]



\section{Simplification via Symmetries, Nondimensionalization, and Dominated Actions}
The way we have defined the problem above is clearly overparameterized.  First
we observe several symmetries\jnote{actually all just facets of Galilean symmetry}:
\begin{itemize}
\item Symmetry under fixed translation (absolute position doesn't matter; only the relative
  position $\vec{r} \eqdef \vec{x}_E - \vec{x}_P$);
\item Symmetry under constant motion (absolute velocities don't matter; only
  the relative velocity $\vec{v} \eqdef \dot{\vec{x}}_E - \dot{\vec{x}}_P$);
\item Isotropy (absolute directions don't matter; only the relative direction
  $\theta \eqdef \arccos(\frac{\vec{r}}{\| \vec{r} \|} \cdot \frac{\vec{v}}{\|
    \vec{v}\|})$).
\end{itemize}

There is one less-obvious symmetry that follows from dimensional analysis (we
will confirm it explicitly later when we derive the system dynamics).  We are
interested in the functions $\vec{a}_P$ and $\vec{a}_E$, which output
accelerations (units $[m \cdot s^{-2}]$) given inputs whose units are either
distances ($\vec{r}$ has units $[m]$) or velocities ($\vec{v}$ has units
$[m \cdot s^{-1}]$).  Let's write $\vec{a} = a \hat{\vec{a}}$,
$\vec{r} = r \hat{\vec{r}}$, and $\vec{v} = v \hat{\vec{v}}$, where $\hat{\vec{a}}$,
$\hat{\vec{r}}$, and $\hat{\vec{v}}$ are dimensionless unit vectors, and $a$, $r$, and $v$ are scalars that respectively have dimensions of
acceleration, length, and speed.

Up to dimensionless constant factors, there are two independent ways we can
construct a scalar with units of acceleration: $A$ and $v^2 / r$.  This means
that $\vec{a}$.

Up to a dimensionless constant multiplicative factor, $\hat{\vec{a}}$
must depend only on $\hat{\vec{r}}$ and $\hat{\vec{v}}$ (specifically $\theta$).


We also observe that without loss of generality $P$ and $E$ both perform
only ``in-plane'' burns, i.e. it should hold that $\ddot{\vec{x}}_P \cdot (\vec{r} \times \vec{v})
= 0$ and $\ddot{\vec{x}}_E \cdot ( \vec{r} \times \vec{v}) = 0$.  \jnote{todo:
  elaborate}
My intuition for this is based on the following idea: the
  differential game should be the limit as $dt \to 0$ of a discrete game in which the
  following steps are cycled indefinitely:
  \begin{enumerate}
  \item Pursuer sets its own acceleration $\vec{a}_P$ as a function of
    $\vec{x}_P$, $\vec{v}_P$, $\vec{x}_E$, and $\vec{v}_E$.
  \item Environment first updates
    $\vec{v}_P \gets \vec{v}_P + \vec{a}_P \cdot dt$ and ``charges'' the pursuer
    $\| \vec{a}_P \| \cdot dt$.  Then, if $\vec{x}_E$ lies on the line segment
    connecting $\vec{x}_P$ and $\vec{x}_P + \vec{v}_P \cdot dt$, the pursuer is
    said to ``capture'' the evader and the game terminates.  Otherwise, the
    environment updates $\vec{x}_P \gets \vec{x}_P + \vec{v}_p \cdot dt$.
  \item Evader sets its own accelerations $\vec{a}_E$ as a function of
    $\vec{x}_P$, $\vec{v}_P$, $\vec{x}_E$, and $\vec{v}_E$.
  \item Environment ``forward steps'' the positions and velocities of the evader
    by updating
    $\vec{v}_E \gets \vec{v}_E + \vec{a}_E \cdot dt$ and then
    $\vec{x}_E \gets \vec{x}_E + \vec{v}_E \cdot dt$.
  \end{enumerate}

  At each state, there is a finite \jnote{and well-defined?} number of steps until capture with
  optimal play.  The WLOG claim follows by induction on this number.  If the
  prover can capture within a single step, it can certainly do so with an
  in-plane burn.  \jnote{todo: continue the argument...or actually maybe there's
  some way to make it trivial using the fact that the end condition can be
  defined just in terms of in-plane parameters...and that the set of reachable
  in-plane parameters by either party is largest with in-plane steps}

\section{Formulation as a differential game}

We formulate the problem as a differential game and employ the
Hamilton--Jacobi--Isaacs (HJI) equation to obtain a differential equation constraining
the value function.

\subsection{The HJI equation}

Let us introduce some notation:
\begin{enumerate}
\item $\vec{s}$ is a vector representing the game state, whose elements are
$(s_1, \dots, s_n)$
\item $t$ is time
\item $\phi$ is the pursuer's control
\item $\psi$ is the evader's control
\item $\nabla$ is the vector of partial derivatives $\left(\pdv{}{s_1}, \dots,
\pdv{}{s_n}\right)$
\item $V(\vec{s}, t)$ is the value function of the game
\item $F(\vec{s}, t, \phi, \psi)$ is a function representing the system dynamics
\item $G(\vec{s}, t, \phi, \psi)$ is the cost rate function
\item $H(\vec{s}, t)$ is the bequest value
\end{enumerate}

$F$ represents the dynamics of the system via the first-order differential
equation:
\begin{equation}
\dot{\vec{s}} = F(\vec{s}, \phi, \psi)
\end{equation}

The HJI equation may be written as follows:
\begin{align}
\pdv{V(\vec{s}, t)}{t} + \min_{\phi} \max_{\psi} \left[
  \nabla V(\vec{s}, t) \cdot F(\vec{s}, \phi, \psi) + G(\vec{s}, \phi, \psi)
\right]
\end{align}
subject to the boundary condition $V(\vec{s}, t) = H(\vec{s}, t)$ within the
capture set.
TODO more considerations for capture formalism may exist in general?

\subsection{Specializing to the 1M1S game}

We specialize the HJI equation for the case of the missile pursuit game.
Here, $V$, $F$, and $G$ lack time dependence, and $H$ is zero.

As previously established, the state space $\vec{s}$ is three-dimensional:
\begin{align}
\vec{s} = (\rho, \sigma, \tau)
\end{align}
where $\rho$ is the distance, $\sigma$ is the radial velocity, and $\tau$ is the
tangential velocity.

The system dynamics are: \jnote{I think this accounts for centrifugal and }
\begin{align}
\dot{\vec{s}}
&= F(\rho, \sigma, \tau, \vec{a}_m, \vec{a}_s) \notag \\
&= \left(\sigma,
  \frac{\tau^2}{\rho} + a_{m\sigma} - a_{s\sigma},
  -\frac{\sigma\tau}{\rho} + a_{m\tau} - a_{s\tau}\right)
\end{align}

We will also write $\vec{a}_m = \phi$ for the missile's control and $\vec{a}_s =
\psi$ for the spacecraft's control.

Substituting all of this, the HJI equation for this game is:
\begin{align}
\min_{\vec{a}_m} \max_{\vec{a}_s} \left[
  \pdv{V}{\rho} \sigma
+ \pdv{V}{\sigma} \left(\frac{\tau^2}{\rho} + a_{m\sigma} - a_{s\sigma}\right)
+ \pdv{V}{\tau} \left(-\frac{\sigma\tau}{\rho} + a_{m\tau} - a_{s\tau}\right)
+ ||\vec{a}_m||
\right] = 0
\end{align}

I am not really sure how to solve something that looks like that.
The reason the existence of $\pdv{V}{x}$ terms is problematic for a numerical
solution is because $\dot{x}$ can be very high: the missile acceleration may be
effectively infinite.
Then the Taylor series approximation used in the HJB equation derivation breaks
down.
That is, in our case, $V(x(t + dt))$ is not well-approximated as $V(x) +
\pdv{V}{x} \dot{x} dt$.
Fortunately, we can actually work with $V(x(t + dt))$ directly, using Bellman's
original \textit{discrete} principle of optimality.

The idea is that for a given state $\vec{s}$, the value after a single timestep
had better be the minmax value over states accessible after a single timestep.
\begin{align}
V(\vec{s}) = \min_{\vec{a}_m} \max_{\vec{a}_s} \left[
  V(\vec{s}_\text{next}(\vec{a}_m, \vec{a}_t, dt)) +
  ||\vec{a}_m||
\right]
\end{align}

Here, $\vec{s}_\text{next}$ is basically just $\vec{s} + F(\dots) dt$.

\section{Analysis of acceleration matching}

Acceleration matching is an intuitive strategy that has been proposed and
analyzed previously. Here we recapitulate the analysis. We repeat the derivation
of the optimal allocation of $\Delta$v between the boost and terminal stages,
and the maximum effective range of this missile assuming a $\Delta$v limit.

In the acceleration matching strategy, missile behavior has two stages:
\begin{enumerate}
\item Boost phase, in which the missile instantaneously gains some velocity
$v_\text{boost}$ toward the target.
\item Terminal phase, in which the missile accelerates to match target
accelerations. In this stage the missile has some $\Delta$v budget
$v_\text{term}$.
\end{enumerate}

Consider the situation where the missile and target begin at time $t = 0$ at a
distance $D$ apart and 0 relative velocity. Assume the target has an
acceleration capacity $a$.

In acceleration matching, during the terminal phase, the missile maintains a
constant closure velocity $v_\text{boost}$, since it thrusts to cancel the
target acceleration. The time to reach the target $t_\text{hit}$ is given by:
\begin{equation}
t_\text{hit} = \frac{D}{v_\text{boost}}
\end{equation}

The missile is defeated if it runs out of $\Delta$v before reaching the target.
If the target maintains full acceleration $a$ in any direction, the missile
$\Delta$v budget runs out in time $t_\text{defeat}$ given by:
\begin{equation}
t_\text{defeat} = \frac{v_\text{term}}{a}
\end{equation}

In order to hit, we require $t_\text{hit} \le t_\text{defeat}$. Using this, we
can derive the maximum distance $D$ at which the missile may hit the target
given $v_\text{boost}$, $v_\text{term}$, and $a$:
\begin{align}
\frac{D}{v_\text{boost}} &\le \frac{v_\text{term}}{a} \\ \notag
D &\le \frac{v_\text{term} v_\text{boost}}{a}
\end{align}

Given a total $\Delta$v budget $v = v_\text{boost} + v_\text{term}$, we may
select an apportionment that maximizes $D$ by maximizing a quadratic:
\begin{align}
v_\text{boost}^* &= \argmax_{v_\text{boost}}
    \frac{v_\text{boost} (v - v_\text{boost})}{a} \\
v_\text{boost}^* &= \frac{v}{2}
\end{align}

Substituting, we obtain the maximum hit distance $D^*$ for a given total
$\Delta$v budget $v$:
\begin{equation}
D^* = \frac{v^2}{4a}
\end{equation}

\section{Lateral acceleration matching is suboptimal}

We give a simple argument that accelerating longitudinally in the terminal phase
leads to a longer effective range.

\balance

\newpage

\bibliography{paper}

\end{document}

%%% Local Variables:
%%% mode: latex
%%% TeX-master: t
%%% End:
